%フォーマット更新日:20210728
%
%

\documentclass[10pt,onecolumn]{jsarticle}

\usepackage[dvipdfmx]{graphicx}
\usepackage{multirow}
\usepackage{url}
\usepackage{otf}
\usepackage{here}
\usepackage{bm}
\usepackage{amsmath}
\usepackage{algorithmic}
\usepackage{algorithm}

\renewcommand{\refname}{次に読むべき論文のリスト}


\newcommand{\hama}{\ajMayuHama}


\pagestyle{empty}

\setlength{\topmargin}{6mm}
\setlength{\oddsidemargin}{-4mm}
\setlength{\evensidemargin}{-4mm}
\setlength{\textwidth}{175mm}
\setlength{\headsep}{0pt}
\setlength{\headheight}{0pt}
\setlength{\textheight}{235mm}
\setlength{\columnsep}{5mm}

\begin{document}

%\twocolumn[%
\vspace{-20mm}
\begin{center}
{\LARGE\textbf{論文メモ}}
\end{center}

\begin{flushright}
\begin{tabular}{|c|l|}
%\hline
%版数  &   0001からはじめる
%\\
\hline
文献番号  &  0014
\\
\hline
日付  &  2021年12月13日
\\
\hline
名前  &  武川海斗
\\
\hline
\end{tabular}
\end{flushright}
%]

%--------------
%本文開始
%--------------

%-------------------------------------------------------------------------
%\section*{論文情報}
%-------------------------------------------------------------------------
%
%論文の基本情報についてまとめる
%
\begin{center}
{\large 文献情報}
\begin{table}[hbp]%[H]
\begin{tabular}{|l||l|}
\hline
著者  &  Motonobu Kanagawa, Philipp Hennig,  Dino     Sejdinovic, and Bharath K Sriperumbudur
\\ \hline
英文タイトル  &  Gaussian Processes and Kernel Methods: A Review  on Connentions and Equivalences
\\ \hline
和文タイトル  & ガウス過程とカーネル法: 接続と等価性に関するレビュー
\\ \hline
書誌情報  &  arXiv:18107.02582v1[stat.ML],2019
\\ \hline
キーワード & なし
\\ \hline
\end{tabular}
\end{table}
\end{center}


\section{論文の要約}
本論文は,ガウス過程とカーネル法の関係性についての研究をまとめた解説論文である.ガウス過程はノンパラメトリックな手法であるが,ベイズ統計学に基づいて,分散についてカーネル関数で表現する.一方,カーネル法とは,入力データを高次元に写像し,再生核ヒルベルト空間で内積計算を行うことで,次元の呪いの影響を受けない手法である.両者にはカーネル関数を通じて,密接な関係がある.特に回帰に関して,ガウス過程回帰とカーネルリッジ回帰に関して,等価性を持つことが知られている.
しかし,ガウス過程では再生核ヒルベルト空間を通じてカーネルを扱わない点に関して,両者は異なるという意見がある.本論文のメッセージとして,確かに,同様の空間上で計算を行わないが,ガウス過程における計算空間も,再生核ヒルベルト空間の累乗として存在することを主張している.

\section{ガウス過程の定義}
本論文では,ガウス過程の定義について触れている.卒業論文においても引用した箇所であるため,本報告書においてもまとめを行う.以下に定義とその解釈について述べる.

空ではない$\bm{\chi}$が存在し,平均$m: \bm{\chi} \to \Re $,正定値性を持つ共分散行列$K:\bm{\chi}\times \bm{\chi}\to \Re$が存在する.
このとき,$m$と$K$を要素にもつランダムな関数$f:\bm{\chi} \to  \Re$をガウス過程と呼び,$\text{GP}(m,K)$と表記する.ここで,$n\in N$の有限集合$\bm{X} = (x_1,x_2,\cdots,x_n) \subset \bm{\chi}$に対して,$\bm{f}_x = \left( \bm{f}(x_1),\bm{f}(x_2),\cdots,\bm{f}(x_n)\right) \in \Re^{n}$
は多変量正規分布$\text{N}(\bm{m}(\bm{x}),\bm{K})$に従う.ただし,
\begin{align}
	\bm{m}(\bm{x}) = \left( m(x_1),m(x_2),\cdots, m(x_n)\right)^T
\end{align}
\begin{align}
	\bm{K} = \left\{k(x_i,x_j) \mid i = 1\sim n,j=1\sim n \right\}
\end{align}
に従う.

ここで,$\bm{f}$がガウス過程である場合,平均関数$\bm{m(x)}$と正定値カーネル$\bm{K}$が存在することを意味する.逆に,平均関数$\bm{m}$,正定値カーネル$\bm{K}$が存在する場合,対応するガウス過程$\bm{f}\sim \text{GP}(\bm{m},\bm{K})$が存在する.このように,ガウス過程$\bm{f}\sim \text{GP}(\bm{m},\bm{K})$
と、平均関数$\bm{m}$と正定値カーネル$\bm{K}$のペアには、一対一の対応関係がある.
そのため,入力データが観測されれば,$\bm{f}$の事後分布を求めることができ,ガウス過程はベイズ統計の立場から見たカーネル法であると言える.

\section{特に重要な関連研究}
本論文は,ガウス過程とカーネル法との関連性についてまとめた論文であり,新規手法については提案されていない.従って,本節では,今後の研究で重要となる論文について紹介する.

\subsection{A Hilbert space embedding for distributions}
一般に,確率分布に対してカーネル法を用いる場合,相互情報量、エントロピー、カルバック・ライブラー発散などの量を計算する必要がある.
そこで,参考文献\cite{ref1}では,こういった中間密度推定を必要とせずに,分布間の距離を推定する手法について提案を行っている.この理論を考える上で重要となるのが,再生核ヒルベルト空間において,確率分布をどのように扱うかという問題であり,本論文ではその紹介と説明を行っている.
確率分布とカーネル法との間の橋渡しを行っている論文であり,今後重要であると言える.

\subsection{Kernel measures of conditional dependence}
参考文献\cite{ref2}は,再生カーネルヒルベルト空間上の正規化交差共分散演算子に基づく,確率変数の条件付き依存性の新しい尺度の提案を行っている.この依存性を,客観的な尺度で評価することは,カーネル法に拡張できるのかを評価できることにつながる.

\begin{thebibliography}{99}
%
\bibitem{ref1}
A. Smola, A. Gretton, L. Song, and B. Scholkopf. A Hilbert space embedding for distributions. \textit{In Proceedings of the International Conference on Algorithmic Learning Theory}, volume 4754, pp.~13-–31. Springer, 2007.

\bibitem{ref2}
K. Fukumizu, A. Gretton, X. Sun, and B. Scholkopf. Kernel measures of conditional dependence.
\textit{In Advances in Neural Information Processing Systems 20}, pp.~489-–496, 2008.


%
\end{thebibliography}



%%%%%%%%%%%%%%%%%%%%
%%%%%%%%%%%%%%%%%%
\end{document}
