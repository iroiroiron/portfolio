%フォーマット更新日:20210728
%
%

\documentclass[10pt,onecolumn]{jsarticle}

\usepackage[dvipdfmx]{graphicx}
\usepackage{multirow}
\usepackage{url}
\usepackage{otf}
\usepackage{here}
\usepackage{bm}
\usepackage{amsmath}
\usepackage{algorithmic}
\usepackage{algorithm}

\renewcommand{\refname}{次に読むべき論文のリスト}


\newcommand{\hama}{\ajMayuHama}


\pagestyle{empty}

\setlength{\topmargin}{6mm}
\setlength{\oddsidemargin}{-4mm}
\setlength{\evensidemargin}{-4mm}
\setlength{\textwidth}{175mm}
\setlength{\headsep}{0pt}
\setlength{\headheight}{0pt}
\setlength{\textheight}{235mm}
\setlength{\columnsep}{5mm}

\begin{document}

%\twocolumn[%
\vspace{-20mm}
\begin{center}
{\LARGE\textbf{論文メモ}}
\end{center}

\begin{flushright}
\begin{tabular}{|c|l|}
%\hline
%版数  &   0001からはじめる
%\\
\hline
文献番号  &  0012
\\
\hline
日付  &  2021年12月07日
\\
\hline
名前  &  武川海斗
\\
\hline
\end{tabular}
\end{flushright}
%]

%--------------
%本文開始
%--------------

%-------------------------------------------------------------------------
%\section*{論文情報}
%-------------------------------------------------------------------------
%
%論文の基本情報についてまとめる
%
\begin{center}
{\large 文献情報}
\begin{table}[hbp]%[H]
\begin{tabular}{|l||l|}
\hline
著者  &  T.~Chai and R.~R.~DraxIer
\\ \hline
英文タイトル  &
\begin{tabular}{l}

Root mean square error (RMSE) or mean absolute error (MAE)?\\ – Arguments against avoiding RMSE in the literature

\end{tabular}

\\ \hline
和文タイトル  &

\begin{tabular}{l}
	二乗平均平方根誤差(RMSE)か平均絶対誤差(MAE)か?\\
	- RMSEを避けることに対する文献上の議論
\end{tabular}
\\ \hline
書誌情報 & Geoscientific Model Development,pp.~1247--1250,2014
\\ \hline
キーワード &
なし.
\\ \hline
\end{tabular}
\end{table}
\end{center}

\section{論文の要約}
RMSEとMAEは誤差情報を基に,モデル評価の指標によく使われる.本論文では,RMSEとMAEの指標の違いについて論じている.先に結論として,どちらが優れているというものではない.RMSEとMAE複数の測定基準を組み合わせることが必要な場合があります。

\section{ベースとなった手法}
以下の式は,MAE,RMSEの定義式である.2式の明確な違いは,絶対誤差か二乗誤差かである.MAEは絶対誤差を扱うのだが,これは最適化の際に問題が起こりやすい.一般に二乗数の方が最適化が容易であり,最適化の面ではRMSEの方が有用である.

\begin{align}
	\mathrm{MAE}=\frac{1}{n} \sum_{i=1}^{n}\left|e_{i}\right|
\end{align}

\begin{align}
	\mathrm{RMSE}=\sqrt{\frac{1}{n} \sum_{i=1}^{n} e_{i}^{2}}
\end{align}
\section{実験デザイン・結果と考察}
RMSEとMAEの性能を比較するために,ランダムに生成したデータを基に実験を行っている.データ数nを$4,10,100,...,1000000$まで試した結果を,表としてまとめている.RMSEでは,nが100以上になると誤差分布が5\%以内になるため,「正確な解」に収束している.逆にデータ数が100未満の場合,RMSEの誤差が大きく,信頼性が低いと言える.そのため,データ数さえ多ければ信頼性の高い結果を得ることができる.

一方,MAEの場合は,データ数を増やした場合,0.8に収束している.やはり,MAEの場合もデータ数の量によって信頼性が変わることを本論文では述べている.

しかし,これらのランダムデータは,仮定したガウス分布から生成したものである.RMSEは,二乗誤差を扱うため,極端に外れた値が含まれている場合,過度に反応してしまい,ロバスト性に欠ける.一方,MAEは絶対誤差を扱うため,全てのデータについて平等な影響を受ける.そのため,ロバスト性という面では優れていると言える.しかし,RMSEはMAEと異なり,データごとに「重み」づけすることでモデル性能指標に違いをつけることができる面で,有用であると言える.

これらの議論から,RMSEとMAEのどちらが優れているということはない.そのため,RMSEとMAEを含めたさまざまな評価指標が今後登場することが期待される.




%%%%%%%%%%%%%%%%%%%%
%%%%%%%%%%%%%%%%%%
\end{document}
